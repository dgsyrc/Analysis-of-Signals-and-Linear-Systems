\section{信号分解为正交函数}

\subsection{矢量正交与正交分解}

\begin{BoxDefinition}[矢量正交]
    矢量正交指矢量$V_x=(v_{x1},v_{x2},v_{x3})$与$V_y=(v_{y1},v_{y2},v_{y3})$的内积为零,即
    \begin{Equation}
        \boldsymbol{V}_x\boldsymbol{V}_y^{T} = \sum\limits_{i=1}^{3}v_{xi}v_{yi} = 0
    \end{Equation}
\end{BoxDefinition}

\begin{BoxDefinition}[矢量正交集与正交分解]
    矢量正交集指由两两正交的矢量组成的矢量集合。

    正交分解即任意向量用矢量正交集中的向量表示。
\end{BoxDefinition}

\subsection{信号正交与正交函数集}

\begin{BoxDefinition}[信号正交]
    定义在$(t_1,t_2)$区间的$\varphi_1(t)$和$\varphi_2(t)$满足
    \begin{Equation}
        \int_{t_1}^{t_2}\varphi_1(t)\varphi_2(t)dt=0
    \end{Equation}
    则称$\varphi_1(t)$和$\varphi_2(t)$在区间$(t_1,t_2)$正交。

    可简记为函数内积为$0$。
\end{BoxDefinition}

\begin{BoxDefinition}[正交函数集]*
    若$n$个函数$\varphi_1(t),\varphi_2(t),\dots,\varphi_n(t)$构成一个函数集,这些函数在区间$(t_1,t_2)$满足
    \begin{Equation}
        \int_{t_1}^{t_2}\varphi_i(t)\varphi_j(t)dt = \left\{\begin{aligned}
            0 & , & i\neq j    \\
            K_i \neq 0 & , & i = j
        \end{aligned} 
    \right.
    \end{Equation}
    则称此函数集为在区间$(t_1,t_2)$的正交函数集。

    若该正交函数集之外不存在函数与集合内函数正交,则称此函数集为完备正交函数集。

    常见的两个在区间$(t_0,t_0+T)(T=\frac{2\pi}{\Omega})$上的完备正交函数集:

    三角函数集
    \begin{Equation}
        \left\{1,\cos(n\Omega t),\sin(n\Omega t), n=1,2,\dots\right\}
    \end{Equation}
    虚指数函数集
    \begin{Equation}
        \left\{e^{\mathrm{j}n\Omega t},n=0,\pm 1,\pm 2,\dots \right\}
    \end{Equation}
\end{BoxDefinition}

\subsection{信号的正交分解}

\begin{BoxDefinition}[信号的正交分解]*
    设有$n$个函数$\varphi_1(t),\varphi_2(t),\dots,\varphi_n(t)$在区间$(t_1,t_2)$构成一个正交函数空间。将任一函数$f(t)$用这$n$个正交函数的线性组合来近似,可表示为
    \begin{Equation}
        f(t)\approx C_1\varphi_1(t)+C_2\varphi_2(t)+\dots+\varphi_n(t)
    \end{Equation}
    当近似函数与原函数的均方误差为$0$时
    \begin{Equation}
        f(t) = \sum\limits_{i=1}^{\infty} C_i\varphi_i(t)
    \end{Equation}
    \begin{Equation}
        C_i = \frac{1}{K_i}\int_{t_1}^{t_2}f(t)\varphi_i(t)dt
    \end{Equation}
    \begin{Equation}
        K_i = \int_{t_1}^{t_2} \varphi_i^2(t)dt
    \end{Equation}
\end{BoxDefinition}

\begin{BoxFormula}[巴塞瓦尔能量公式]*
    巴塞瓦尔能量公式
    \begin{Equation}
        \int_{t_1}^{t_2} f^2(t) dt= \sum\limits_{i=1}^{\infty}C_i^2K_i
    \end{Equation}
    表示在区间$(t_1,t_2)$上$f(t)$所含能量恒等于$f(t)$在完备正交函数集中分解的各正交分量能量之和。
\end{BoxFormula}

