\section{拉普拉斯变换性质}

\subsection{拉氏变换的线性性质}

\begin{BoxProperty}[拉普拉斯变换的线性性质]
    若$f_1(t)\longleftrightarrow F_1(s)\quad\mathrm{Re}\left[s\right]>\sigma_1$,$f_2(t)\longleftrightarrow F_2(s)\quad \mathrm{Re}\left[s\right]>\sigma_2$,则
    \begin{Equation}
        a_1f_1(t)+a_2f_2(t)\longleftrightarrow a_1F_1(s)+a_2F_2(s) \quad \mathrm{Re}\left[s\right]>\max(\sigma_1,\sigma_2)
    \end{Equation}
\end{BoxProperty}

\subsection{拉氏变换的尺度变换}
\begin{BoxProperty}[拉普拉斯变换的尺度变换]
    若$f(t)\longleftrightarrow F(s)\quad\mathrm{Re}\left[s\right]>\sigma_0$,且$a>0$,则
    \begin{Equation}
        f(at)\longleftrightarrow\frac{1}{a}F(\frac{s}{a})
    \end{Equation}
\end{BoxProperty}
    
\subsection{拉氏变换的时移特性}

\begin{BoxProperty}[拉普拉斯变换的时移特性]
    若$f(t)\longleftrightarrow F(s)\quad\mathrm{Re}\left[s\right]>\sigma_0$,且$t_0>0$,则
    \begin{Equation}
        f(t-t_0)\varepsilon(t-t_0)\longleftrightarrow e^{-st_0}F(s)\quad\mathrm{Re}\left[s\right]>\sigma_0
    \end{Equation}
    与尺度变换结合
    \begin{Equation}
        f(at-t_0)\varepsilon(at-t_0)\longleftrightarrow \frac{1}{a}e^{-\frac{t_0}{a}s}F(\frac{s}{a})
    \end{Equation}
\end{BoxProperty}

\subsection{拉氏变换的复频移特性}

\begin{BoxProperty}[拉普拉斯变换的复频移特性]
    若$f(t)\longleftrightarrow F(s)\quad\mathrm{Re}\left[s\right]>\sigma_0$,且$s_a = \sigma_a + \mathrm{j}\omega_a$,则
    \begin{Equation}
        f(t)e^{s_a t}\longleftrightarrow F(s-s_a)\quad\mathrm{Re}\left[s\right]>\sigma_0+\sigma_a
    \end{Equation}
\end{BoxProperty}

\subsection{拉氏变换的时域的微分特性}

\begin{BoxTheorem}[拉普拉斯变换的微分定理]
    若$f(t)\longleftrightarrow F(s)\quad\mathrm{Re}\left[s\right]>\sigma_0$,则
    \begin{Equation}
        f'(t)\longleftrightarrow sF(s)-f(0_{-})
    \end{Equation}
    进一步推导可得
    \begin{Equation}
        f^{(n)}(t) \longleftrightarrow s^nF(s)-\sum\limits_{m=0}^{n-1}s^{n-1-m}f^{(m)}(0_{-})
    \end{Equation}
    若$f(t)$为因果函数
    \begin{Equation}
        f^{(n)}(t) \longleftrightarrow s^nF(s) \quad \mathrm{Re}\left[s\right] > \sigma_0
    \end{Equation}
\end{BoxTheorem}

\subsection{拉氏变换的时域积分特性}

\begin{BoxTheorem}[拉普拉斯变换的积分定理]
    若$\mathscr{L}\left[f(t)\right]= F(s)$,则
    \begin{Equation}
        \mathscr{L}\left[\int_{-\infty}^{t}f(\tau)d\tau\right] = \frac{F(s)}{s}+\frac{f^{(-1)}(0_{-})}{s}
    \end{Equation}
    若$f(t)$为因果信号,且$f^{(n)}(0_{-})=0$,则
    \begin{Equation}
        \left(\int_{0_{-}}^{t}\right)^{n}f(x)dx \longleftrightarrow \frac{F(s)}{s^{n}}
    \end{Equation}
    同理,若$f(t)$为因果信号,若$f^{(n)}(t)\longleftrightarrow F_{n}(s)$,则
    \begin{Equation}
        f(t) \longleftrightarrow \frac{F_n(s)}{s^n}
    \end{Equation}
\end{BoxTheorem}

\subsection{拉氏变换的卷积定理}

\begin{BoxTheorem}[拉普拉斯变换的时域卷积定理]
    若因果函数$f_1(t)\longleftrightarrow F_1(s)\quad \mathrm{Re}\left[s\right]>\sigma_1$, $f_2(t)\longleftrightarrow F_2(s) \quad \mathrm{Re}\left[s\right]>\sigma_2$,则
    \begin{Equation}
        f_1(t)*f_2(t) \longleftrightarrow F_1(s)F_2(s)
    \end{Equation}
\end{BoxTheorem}

\begin{BoxTheorem}[拉普拉斯变换的复频域卷积定理]
    若因果函数$f_1(t)\longleftrightarrow F_1(s)\quad \mathrm{Re}\left[s\right]>\sigma_1$, $f_2(t)\longleftrightarrow F_2(s) \quad \mathrm{Re}\left[s\right]>\sigma_2$,则
    \begin{Equation}
        f_1(t)f_2(t)\longleftrightarrow\frac{1}{2\pi\mathrm{j}}\int_{c-\mathrm{j}\infty}^{c+\mathrm{j}\infty} F_1(\eta)F_2(s-\eta)d\eta
    \end{Equation}
\end{BoxTheorem}

\subsection{s域微分和积分定理}

\begin{BoxTheorem}[s域微分和积分定理]
    若$f(t)\longleftrightarrow F(s)$,$\mathrm{Re}\left[s\right]>\sigma_0$,则
    \begin{Equation}
        (-t)^{n}f(t)\longleftrightarrow\frac{d^{n}F(s)}{ds}
    \end{Equation}
    特别地
    \begin{Equation}
        \frac{f(t)}{t}\longleftrightarrow\int_{s}^{\infty}F(\eta)d\eta
    \end{Equation}
\end{BoxTheorem}

\subsection{初值定理和终值定理}

\begin{BoxTheorem}[初值定理]
    设函数$f(t)$不含$\delta(t)$及其各阶导数(即$F(s)$为真分式,若$F(s)$为假分式则化为真分式)
    \begin{Equation}
        f(0_{+}) = \lim\limits_{t\rightarrow 0_{+}}f(t) = \lim\limits_{s\rightarrow\infty}sF(s)
    \end{Equation}
\end{BoxTheorem}

\begin{BoxTheorem}[终值定理]
    若$f(t)$当$t\rightarrow \infty$时存在,并且$f(t)\longleftrightarrow F(s)$,$Re\left[s\right]>\sigma_0$,$\sigma_0<0$,则
    \begin{Equation}
        f(\infty) = \lim\limits_{s\rightarrow 0 }sF(s)
    \end{Equation}
\end{BoxTheorem}