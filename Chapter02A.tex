\section{LTI连续系统的响应}

\subsection{微分方程的经典解}

LTI连续系统由常系数$n$阶线性常微分方程表示

\begin{BoxDefinition}[常系数$n$阶线性常微分方程]
    常系数$n$阶线性常微分方程
    \begin{Equation}
        \begin{array}{ll}
            y^{n}(t)+a_{n-1}y^{(n-1)}(t)+\dots+a_1y^{(1)}(t)+a_0y(t)=b_mf^{(m)}(t)+b_{m-1}f^{m-1}(t) \\
            +\dots+b_1f^{(1)}(t)+b_0f(t)
        \end{array}
    \end{Equation}
\end{BoxDefinition}

\begin{BoxFormula}[微分方程的齐次解]
    列特征方程
    \begin{Equation}
        \lambda^n+a_{n-1}\lambda^{n-1}+\dots+a_1\lambda+a_0 = 0
    \end{Equation}
    解出特征根,齐次解的形式由特征根确定。

    若为$n$个单实特征根
    \begin{Equation}
        y_h(t) = \sum\limits_{i=1}^{n}c_ie^{\lambda_i t}
    \end{Equation}
    若为$r$重实根
    \begin{Equation}
        y_h(t) = (c_{r-1}t^{r-1}+c_{r-2}t^{r-2}+\dots+c_1t+c_0)e^{\lambda t}
    \end{Equation}
    若为$1$对共轭复根$\lambda_{1,2} = \alpha + \mathrm{j}\beta$
    \begin{Equation}
        y_h(t) = [C\cos(\beta t) + D\sin(\beta t)]e^{\alpha t}
    \end{Equation}
\end{BoxFormula}

齐次解又叫固有响应或自由响应。

\begin{BoxFormula}[微分方程的特解]
    特解的形式由激励函数$f(t)$的形式确定。

    若激励为常数
    \begin{Equation}
        y_p(t) = C
    \end{Equation}
    若激励为$t^m$,且特征根均不为$0$
    \begin{Equation}
        y_p(t) = P_mt^m+P_{m-1}t^{m-1}+\dots+P_1t+P_0
    \end{Equation}
    若激励为$t^m$,且$r$重特征根为$0$
    \begin{Equation}
        y_p(t) = t^r(P_mt^m+P_{m-1}t^{m-1}+\dots+P_1t+P_0)
    \end{Equation}
    若激励为$e^{\alpha t}$,且$\alpha\neq$特征根
    \begin{Equation}
        y_p(t) = Pe^{\alpha t}
    \end{Equation}
    若激励为$e^{\alpha t}$,且$\alpha =$特征根
    \begin{Equation}
        y_p(t) = (P_1t+P_0)e^{\alpha t}
    \end{Equation}
    若激励为$e^{\alpha t}$,且$\alpha = r$重特征根
    \begin{Equation}
        y_p(t) = (P_re^r+P_{r-1}e^{r-1}+\dots+P_0)e^{\alpha t}
    \end{Equation}
    若激励为$\cos(\beta t),\sin(\beta t)$,且特征根$\neq \pm\mathrm{j}\beta$
    \begin{Equation}
        y_p(t) = P_1\cos(\beta t)+P_2\sin(\beta t)
    \end{Equation}
    接着将特解代入响应$y(t)$,通过系数比较法列方程解出特解系数
\end{BoxFormula}

特解又叫强迫响应。

\begin{BoxFormula}[微分方程的全解(一)]
    求解微分方程的全解步骤如下:

    由\xref{fml:微分方程的齐次解}及\xref{fml:微分方程的特解}可得齐次解的形式以及特解

    全解可用下式表示
    \begin{Equation}
        y(t) = y_h(t) + y_p(t)
    \end{Equation}
    列出全解后根据初始条件($y(0),y'(0)$的值)列出方程组解出系数即可

    注意全解的定义域范围为$t\geq0$,最后写出的答案乘上$\varepsilon(t)$或写出$t$的取值范围即可
\end{BoxFormula}

\subsection{关于$0_{-}和0_{+}$状态的转换}

\begin{BoxProperty}[冲激函数匹配法]
    在$t=0$时刻,根据激励的冲激函数及其系数确定响应的的冲激函数系数,且左右两边冲激函数各阶导数系数平衡相等。

    如$y'(t)+3y(t)=3\delta'(t)$,由于方程右边有$3\delta'(t)$,故左边$y'(t)$含$3\delta'(t)$,$y(t)$含$3\delta(t)$,但由于右边不存在$3\delta(t)$,故$y'(t)$包含$-9\delta(t)$抵消$3y(t)$中的$9\delta(t)$,此时$y(t)$含有$-9\varepsilon(t)$,冲激函数及其各阶导数系数平衡相等,故可得$t=0$时$y(t)$跳变,且$y(0_+)-y(0_-)=-9$
\end{BoxProperty}

\subsection{零输入响应和零状态响应}

\begin{BoxFormula}[微分方程的全解(二)]
    微分方程的全解还可以用零输入响应和零状态响应表示
    \begin{Equation}
        y(t) = y_{zi}(t) + y_{zs}(t) \quad (t\geq 0)
    \end{Equation}
\end{BoxFormula}


\begin{BoxFormula}[零输入响应]
    零输入响应求解步骤同\xref{fml:微分方程的齐次解}类似,先列微分方程的特征方程解特征根,根据特征根确定解的形式。

    例如特征方程有$n$个单特征实根时
    \begin{Equation}
        y_{zi}(t) = \sum\limits_{j=1}^{n}C_{zij}e^{\lambda_j t}
    \end{Equation}

    列出零输入响应后,代入初始条件解出系数

    初始条件

    \begin{Equation}
        y_{zi}^{(j)}(0_+) = y_{zi}^{(j)}(0_-) = y^{(j)}(0_-)
    \end{Equation}

    $y_{zi}(t)$定义域为$t\geq0$
\end{BoxFormula}

\begin{BoxFormula}[零状态响应]*
    零状态响应解的形式与对应齐次方程通解相似,参考\xref{fml:微分方程的齐次解},区别在于需要加上特解$y_p(t)$,形式的规则同\xref{fml:微分方程的特解}

    例如特征方程有$n$个单特征实根时
    \begin{Equation}
        y_{zs}(t) = \sum\limits_{j=1}^{n}C_{zsj}e^{\lambda_j t}+y_p(t)
    \end{Equation}
    对于通解$y_p(t)$,根据$t>0$时激励$f(t)$的形式确定

    例如

    \begin{Equation}
        f(t) = \varepsilon(t) \quad (t>0) \Rightarrow f(t) = 1 \Rightarrow y_p(t) = C
    \end{Equation}

    零状态响应已有初始条件

    \begin{Equation}
        y_{zs}(0_-) = y_{zs}'(0_-) = 0
    \end{Equation}

    将初始条件及零状态响应代入原方程可解出特解系数

    剩余系数可通过\xref{ppt:冲激函数匹配法}冲激函数匹配法求出$y_{zs}(0_+),y_{zs}'(0_+)$,再代入$t>0$时的方程求解

    冲激函数匹配法过程

    根据激励的冲激项及其系数列出$y_{zs}(t)$各阶导数项,假设激励的冲激项为$a\delta(t)$,响应的最高阶导数为$y''(t)$

    \begin{Equation}
        \left\{
        \begin{array}{ll}
            y_{zs}''(t) = a\delta(t) + r_1(t) \\
            y_{zs}'(t) = r_2(t)               \\
            y_{zs}(t) = r_3(t)
        \end{array}
        \right.
    \end{Equation}

    其中$r_i(t)$为不含$\delta(t)$的某函数

    方程组系数代入原微分方程根据系数平衡解出

    两侧同时从$0_-$到$0_+$积分可得$y_{zs}(0_+),y_{zs}'(0_+)$

    积分满足
    \begin{Equation}
        \begin{array}{ll}
            \int_{0_-}^{0_+}y_{zs}''(t)dt = y_{zs}'(0_+) - y_{zs}'(0_-) \\
            \int_{0_-}^{0_+}r(t)dt = 0                                  \\
            \int_{0_-}^{0_+}a\delta (t)dt = a
        \end{array}
    \end{Equation}
    $y_{zs}(t)$定义域为$t\geq0$
\end{BoxFormula}