\section{能量谱和功率谱}

\subsection{帕斯瓦尔关系}

\begin{BoxDefinition}[帕斯瓦尔关系]
    帕斯瓦尔关系
    \begin{Equation}
        E = \int_{-\infty}^{\infty} |f(t)|^2 dt = \frac{1}{2\pi} \int_{-\infty}^{\infty} |F(\mathrm{j}\omega)|^2 d\omega
    \end{Equation}
\end{BoxDefinition}

\subsection{能量谱密度}

\begin{BoxDefinition}[能量谱密度]
    能量谱指单位频率内信号的能量,记为$E(\omega)$。
    
    在频带$df$内信号的能量为$E(\omega)df$,因而信号在整个频率范围内的总能量

    \begin{Equation}
        E = \int_{-\infty}^{\infty} E(\omega) df = \frac{1}{2\pi} \int_{-\infty}^{\infty} E(\omega) d\omega
    \end{Equation}
    由帕斯瓦尔关系可得
    \begin{Equation}
        E(\omega) = |F(\mathrm{j}\omega)|^2
    \end{Equation}
    即
    \begin{Equation}
        R(\tau) \longleftrightarrow E(\omega)
    \end{Equation}
    能量谱函数与自相关函数是一对傅里叶变换对。
\end{BoxDefinition}

\subsection{功率谱}

\begin{BoxDefinition}[功率谱]*
    功率谱指单位频率的信号功率,记为$P(\mathrm{j}\omega)$。

    在频带$df$内信号的总功率为$P(\omega)df$,因而信号在整个频率范围的总功率
    \begin{Equation}
        P = \int_{-\infty}^{\infty} P(\omega) df = \frac{1}{2\pi} \int_{-\infty}^{\infty} P(\omega) d\omega
    \end{Equation}
    因此
    \begin{Equation}
        P(\omega)  = \lim\limits_{T\rightarrow\infty}\frac{|F_T(\mathrm{j}\omega)|^2}{T}
    \end{Equation}
    即
    \begin{Equation}
        R(\tau) \longleftrightarrow P(\omega)
    \end{Equation}
    功率有限信号的功率谱与自相关函数是一对傅里叶变换对。
\end{BoxDefinition}
